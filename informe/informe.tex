\documentclass[a4paper]{article}
\usepackage[spanish]{babel}
\usepackage[utf8]{inputenc}
\usepackage{charter}   % tipografia
\usepackage{graphicx}
%\usepackage{makeidx}

%\usepackage{float}
%\usepackage{amsmath, amsthm, amssymb}
%\usepackage{amsfonts}
%\usepackage{sectsty}
%\usepackage{charter}
%\usepackage{wrapfig}
%\usepackage{listings}
%\lstset{language=C}


\input{codesnippet}
\input{page.layout}
% \setcounter{secnumdepth}{2}
\usepackage{underscore}
\usepackage{caratulaV}
\usepackage{url}


% ******************************************************** %
%              TEMPLATE DE INFORME ORGA2 v0.1              %
% ******************************************************** %
% ******************************************************** %
%                                                          %
% ALGUNOS PAQUETES REQUERIDOS (EN UBUNTU):                 %
% ========================================
%                                                          %
% texlive-latex-base                                       %
% texlive-latex-recommended                                %
% texlive-fonts-recommended                                %
% texlive-latex-extra?                                     %
% texlive-lang-spanish (en ubuntu 13.10)                   %
% ******************************************************** %



\begin{document}


\thispagestyle{empty}
\materia{Organización del Computador II}
\submateria{Primer Cuatrimestre - 2015}
\titulo{Trabajo Práctico II}
\subtitulo{Filtros de Imagen}
\integrante{Christian Cuneo}{755/13}{chriscuneo93@gmail.com}
\integrante{Ignacio Lebrero}{751/13}{ignaciolebrero@gmail.com}
\integrante{Jorge Porto}{376/11}{cuanto.p.p@gmail.com}

\maketitle
\newpage

\thispagestyle{empty}
\vfill
\begin{abstract}
Los filtros de imagen son una herramienta poderosa a la hora de retocar una imagen, usados ampliamente en fotografia, publicidad, videojuegos, etc.
Su uso brinda una gamma de opciones para modificar las imagenes de manera que sea mas flexible su edicion o analisis.\\
En este trabajo practico presentamos los metodos blur, merge y hsl ya existentes y los implementamos en lenguaje de ensamblador.
Damos dos implementaciones de cada filtro siendo la segunda una optimizacion de la primera en merge y blur, y una variacion de implementacion C/Assembler a Assembler en hsl.\\
Nuestros experimentos demuestran.....
\end{abstract}

\thispagestyle{empty}
\vspace{3cm}
\tableofcontents
\newpage


%\normalsize
\newpage

\section{Objetivos generales}

El objetivo de este Trabajo Práctico es ...


\section{Contexto}

\begin{figure}
  \begin{center}
	\includegraphics[scale=0.66]{imagenes/logouba.jpg}
	\caption{Descripcion de la figura}
	\label{nombreparareferenciar}
  \end{center}
\end{figure}


\paragraph{\textbf{Titulo del parrafo} } Bla bla bla bla.
Esto se muestra en la figura~\ref{nombreparareferenciar}.



\begin{codesnippet}
\begin{verbatim}

struct Pepe {

    ...

};

\end{verbatim}
\end{codesnippet}


\section{Enunciado y solucion} 
\input{enunciado}

\section{Conclusiones y trabajo futuro}


\end{document}

